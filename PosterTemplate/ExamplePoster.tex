\documentclass[final]{beamer}
\mode<presentation>
\usetheme{Example}  %modification of Posterlftuke04

\usepackage{booktabs}
\usepackage{amsfonts}
\usepackage[size=custom,height=86.36,width=111.76,scale=1.3]{beamerposter}
\usepackage{amsmath}
\usepackage{amsthm}
\usepackage{latexsym}
\usepackage{amssymb}
\usepackage{graphicx}
\usepackage{pgf}
\usepackage{pgfplots}
\usetikzlibrary{spy,backgrounds,decorations,positioning, decorations.markings, arrows,shapes, calc}
\tikzset{>=latex}
\usepackage{marvosym}
\usepackage{xcolor}
\usepackage{verbatim}
\usepackage{manfnt}
\usepackage{caption}
%\usepackage{subcaption}
\usepackage{mathtools}
\usepackage{bm}

\DeclarePairedDelimiter\abs{\lvert}{\rvert}


\usepackage{tikz}
\usetikzlibrary{shapes.geometric}
\usetikzlibrary{positioning}
\usetikzlibrary{calc}
\usetikzlibrary{scopes}
\usetikzlibrary{shapes}
\usetikzlibrary{decorations.pathmorphing} 

\usepackage{color}
\definecolor{darkblue}{rgb}{0, 0, .6}
\definecolor{grey}{rgb}{.7, .7, .7}

\selectcolormodel{cmyk}
%\definecolor{naugreen}{cmyk}{.43,0,.34,.38}
%\definecolor{naublue}{cmyk}{1,.72,0,.30}
%\definecolor{softplum}{cmyk}{.02,.45,0,.68}%.37
%\definecolor{orangesalmon}{cmyk}{0,.53,.87,0}
%\definecolor{lightorange}{cmyk}{0,.45,.59,.02}
%\definecolor{brickorange}{cmyk}{0,.59,.59,.20}
%\definecolor{deeppink}{cmyk}{0,.6,.3,.3}
%\definecolor{orchid}{cmyk}{0,.49,.02,.05}
%\definecolor{purple}{RGB}{172,40,246}
%\definecolor{orange}{RGB}{255,102,0}
%\definecolor{springgreen}{cmyk}{1.00,0,1,0}
%
%\definecolor{plum}{RGB}{139,14,58}
%\definecolor{sunshine}{RGB}{252,205,0}
%\definecolor{peony}{RGB}{101,10,97}

%%% Colors %%%
%reds
\definecolor{rred}{rgb}{0.9, 0.17, 0.31}
\definecolor{darkred}{cmyk}{0,1,1,.3}
\definecolor{rose}{cmyk}{0,1.00,.20,0}
\definecolor{brightrose}{cmyk}{0,1.00,.30,0}
\definecolor{bloodorange}{cmyk}{0,.92,.95,.25}

%greens
\definecolor{ggreen}{RGB}{0,153,0}
\definecolor{naugreen}{cmyk}{.64,0,.55,.55}
\definecolor{springgreen}{cmyk}{1.00,0,.70,.02}
\definecolor{limegreen}{RGB}{70,255,25}
\definecolor{lime}{RGB}{	124,252,0}
\definecolor{armygreen}{RGB}{71,115,0}

%blues
\definecolor{darkblue}{RGB}{0,0,255}
\definecolor{turq}{RGB}{72,209,204}
\definecolor{naublue}{cmyk}{1,.72,0,.32}
\definecolor{mediterranean}{cmyk}{.67,0,.08,.3}
\definecolor{butterfly}{cmyk}{.95,.59,0,.10}
\definecolor{icyblue}{cmyk}{.43,0,0.1,0}
\definecolor{lightskyblue}{cmyk}{.4,.11,0,.2}

%orange
\definecolor{orange2}{RGB}{255,100,0}
\definecolor{orange}{RGB}{255,102,0}
\definecolor{nectarine}{cmyk}{0,0.70,1.00,0}

%purples
\definecolor{purple}{RGB}{153,51,255}
\definecolor{darkorchid}{cmyk}{.25,.76,0,.45}
\definecolor{purple2}{RGB}{159,51,250}

%greys
\definecolor{gray}{RGB}{220,220,220}
\definecolor{grey}{RGB}{111,111,111}
\definecolor{manatee}{rgb}{0.59, 0.6, 0.67}
\definecolor{darkgray}{cmyk}{0,0,0,.67}

%officially defined NAU approved colors we can use
\definecolor{sunset}{cmyk}{0,.66,.99,0}%orange
\definecolor{fallaspen}{cmyk}{0,.35,.85,0}%lighterorange
\definecolor{summershade}{RGB}{0,133,63}%green
\definecolor{supai}{RGB}{0,172,165}%bluegreen
\definecolor{sky}{RGB}{0,102,179} %skyblue
\definecolor{twilight}{RGB}{0,102,179} %darker blue
\definecolor{monsoon}{RGB}{195,184,178} %a taupe-ish color
\definecolor{redrock}{RGB}{177,84,29} %brick red/brown
%official colors
\definecolor{nautrueblue}{RGB}{0,51,102}
\definecolor{naugold}{cmyk}{0,.16,1.0,0}



\newcommand{\themename}{\textbf{\textsc{metropolis}}\xspace}
%%----------------%%

\setbeamercolor{alerted text}{fg=sunset}

\DeclareMathOperator{\mex}{mex}
\DeclareMathOperator{\nim}{nim}
\DeclareMathOperator{\Opt}{Opt}

\newtheorem{comments}{Comments}
\newtheorem{question}{Question}
\newtheorem{goal}{Goal}
\newtheorem{remark}{Remark}
\newtheorem{proposition}{Proposition}
\newtheorem{conjecture}{Conjecture}

\newcommand{\w}{\overline{w}}
\newcommand{\uu}{\overline{u}}
\newcommand\xxaxis{0}
\newcommand\yyaxis{90}

\newcommand\altheapblock[4]{
\fill[fill=#4, fill opacity=1, draw=black, line width=1.1pt, rounded corners,shift={(\xxaxis:#1)},shift={(\yyaxis:#2)}] (-.5,-.5) rectangle (.5,.5);\node[inner sep=0pt]at (#1,#2) {\scriptsize $#3$};}

%\renewcommand\thesubfigure{(\alph{subfigure})}

\newcommand\dheapblock[4]{\draw[dotted, draw=#4, line width=1.1pt, rounded corners,shift={(\xxaxis:#1)},shift={(\yyaxis:#2)}] (-.5,-.5) rectangle (.5,.5);\node[inner sep=0pt]at (#1,#2) {\footnotesize $#3$};}

\newcommand\sheapblock[4]{\draw[pattern= north west lines, pattern color=#4, draw=#4, line width=1.1pt, rounded corners,shift={(\xxaxis:#1)},shift={(\yyaxis:#2)}] (-1,-1) rectangle (1,1);\node[inner sep=0pt]at (#1,#2) {\scriptsize $#3$};}

\newcommand\heapblock[4]{\fill[fill=#4, fill opacity=1, draw=black, line width=1.1pt, rounded corners,shift={(\xxaxis:#1)},shift={(\yyaxis:#2)}] (-1,-1) rectangle (1,1);\node at (#1,#2) {\footnotesize $#3$};}


\def\vdotsalt{\vbox{\baselineskip=10pt \lineskiplimit=0pt
\kern6pt \hbox{.}\hbox{.}\hbox{.}}}



\setlength{\fboxrule}{3pt}

\setbeamertemplate{navigation symbols}{}


%%%%%%%%%%%%%%%%%%%%%%%%%%

\title{Structure of braid graphs for Coxeter groups}

\institute{\color{summershade}Department of Mathematics \& Statistics, Northern Arizona University}
\author{Emalina Bidari \& Brandon Samz, Directed by Dana C.~Ernst}

%%%%%%%%%%%%%%%%%%%%%%%%%

\begin{document}

\begin{frame}{}

\vspace{-1em}

\begin{columns}[T]

%%%%%%%%%%%%%%%

\begin{column}{.33\linewidth}
%---------------------%
\begin{block}{Coxeter systems of type $A$ and $B$}

The \alert{Coxeter group $W(B_{n+1})$} is generated by $\{s_0,s_1,s_2,...,s_{n}\}$ and is subject to the defining relations:

\vspace{-0.2em}
\begin{enumerate}
\item $s_is_i = e$ for all $i$
\item $s_is_j = s_js_i$ when $\abs{i-j} > 1$ (\alert{commutation relation})
\item $s_is_{i+1}s_i = s_{i+1}s_is_{i+1}$ if $1\leq i \leq n-1$ (\alert{short braid})\\
\item $s_{0}s_{1}s_{0}s_{1}=s_{1}s_{0}s_{1}s_{0}$ (\alert{long braid})
\end{enumerate}

Removing the generator $s_0$ and the corresponding relation yields the \alert{Coxeter group $W(A_n)$}, which is isomorphic to the symmetric group $S_{n+1}$ via $s_i\mapsto (i,i+1)$. 

\end{block}

%---------------------%

\begin{block}{Reduced expressions}
A word $\w = s_{x_1}s_{x_2}\cdots s_{x_m}$ is called an \alert{expression} for $w\in W$ if it is equal to $w$ when considered as a group element. If $m$ is minimal among all expressions for $w$, then $\w$ is called a \alert{reduced expression}, and the \alert{length} of $w$ is $\ell(w)=m$.
\end{block}

%---------------------%

\begin{block}{Example}
Consider the expression $\w = s_3s_1s_2s_3s_2$ for $w \in W(A_3)$. We see that
\[
s_3s_1\textcolor{limegreen}{s_2s_3s_2}=\textcolor{rose}{s_3s_1}s_3s_2s_3=s_1\textcolor{monsoon}{s_3s_3}s_2s_3=s_1s_2s_3.
\]
Then $s_3s_1s_2s_3s_2$ is not reduced. It turns out that the expression on the right is reduced, so $\ell(w)=3$.
\end{block}

%---------------------%
\begin{block}{Matsumoto's Theorem}
Any two reduced expressions for $w$ differ by a sequence of commutation moves and braid moves.
%\vspace{.08em}
\end{block}

%---------------------%

\begin{block}{Definition}
For $w\in W$, define the \alert{Matsumoto graph} $M(w)$ via:
%\vspace{-.5em}
\begin{itemize}
\item Vertices: reduced expressions for $w$
\item Edges: $\w_1$ and $\w_2$ are connected by an edge iff $\w_1$ and $\w_2$ are related via a single commutation or braid move
\end{itemize}
\end{block}

%---------------------%

\begin{block}{Matsumoto graph for an element in $\mathbf{W(A_3)}$}%{Example of a Matsumoto Graph}

%\vspace{.5em}

%The Matsumoto graph for the ``longest element" in $W(A_3)$:
\begin{figure}
\begin{center}
\includegraphics[scale=3.05]{MastumotoGraph.pdf}
%\begin{tikzpicture}[>=latex, scale=1.4, line join=bevel,]
%\node (node_1) at (5,0)  {\small 121321};
%\node (node_2) at (9,0) {\small 123121};
%\node (node_3) at (3,1.75) {\small 212321};
%\node (node_4) at (11,1.75) {\small 123212};
%\node (node_5) at (2,4) {\small 213231};
%\node (node_6) at (12,4) {\small 132312};
%\node (node_7) at (0,5.5) {\small 231231};
%\node (node_8) at (4,5.5) {\small 213213};
%\node (node_9) at (10,5.5) {\small 312312};
%\node (node_10) at (14,5.5) {\small 132132};
%\node (node_11) at (2,7) {\small 231213};
%\node (node_12) at (12,7) {\small 312132};
%\node (node_13) at (3,9.25) {\small 232123};
%\node (node_14) at (11,9.25) {\small 321232};
%\node (node_15) at (5,11) {\small 323123};
%\node (node_16) at (9,11) {\small 321323};
%
%\draw [red,-,ultra thick] (node_7) to (node_5);
%\draw [red,-,ultra thick] (node_7) to (node_11);
%\draw [red,-,ultra thick] (node_5) to (node_8);
%\draw [red,-,ultra thick] (node_11) to (node_8);
%\draw [red,-,ultra thick] (node_9) to (node_6);
%\draw [red,-,ultra thick] (node_9) to (node_12);
%\draw [red,-,ultra thick] (node_6) to (node_10);
%\draw [red,-,ultra thick] (node_12) to (node_10);
%\draw [red,-,ultra thick] (node_1) to (node_2);
%\draw [red,-,ultra thick] (node_15) to (node_16);
%
%\draw [green,-,ultra thick] (node_1) to (node_3);
%\draw [green,-,ultra thick] (node_2) to (node_4);
%\draw [green,-,ultra thick] (node_3) to (node_5);
%\draw [green,-,ultra thick] (node_4) to (node_6);
%\draw [green,-,ultra thick] (node_11) to (node_13);
%\draw [green,-,ultra thick] (node_12) to (node_14);
%\draw [green,-,ultra thick] (node_13) to (node_15);
%\draw [green,-,ultra thick] (node_14) to (node_16);
%\end{tikzpicture}
\end{center}
%\caption{Matsumoto graph for the ``longest element" in $W(A_3)$}
\end{figure}

\end{block}


%---------------------%

\end{column}

%%%%%%%%%%%%%%%

\begin{column}{.33\linewidth}

%---------------------%

\begin{block}{Braid equivalence}
If $\w_1$ and $\w_2$ are reduced expressions for $w \in W$, then $\w_1$ and $\w_2$ are \alert{braid equivalent} iff we can obtain $\w_2$ from $\w_1$ via a sequence of braid moves. The equivalence classes are called \alert{braid classes}.
\end{block}
%---------------------%

\begin{block}{Example}
Consider the 16 reduced expressions for the an element in $W(A_3)$. Applying all possible braid moves yields 8 braid classes:

\begin{columns}
\begin{column}{0.6\textwidth}
\begin{itemize}
\item[] $[123121]_b = \{123121,123212,132312\}$
\item[] $[312132]_b = \{312132,321232,321323\}$
\item[] $[121321]_b = \{121321,212321,213231\}$
\item[] $[231213]_b = \{231213,232123,323123\}$
\end{itemize}
\end{column}
\begin{column}{0.4\textwidth}
\begin{itemize}
\item[] $[312312]_b = \{312312\}$
\item[] $[132132]_b = \{132132\}$
\item[] $[213213]_b = \{213213\}$
\item[] $[231231]_b = \{231231\}$
\end{itemize}
\end{column}
\end{columns}
\vspace{.2em}
\end{block}

%--------------------%

\begin{block}{Braid chains}
Loosely speaking, a \alert{braid chain} is a reduced expression with the property that we can ``slide" a braid across the whole expression.
\vspace{1em}
\begin{center}
\begin{tikzpicture}[scale=0.6]
\node (label0) at (0,0) {$\bm{\textcolor{rose}{343}}546576$};
\node (label1) at (10,0) {$\bm{\textcolor{supai}{43454}}6576$};
\node(label2) at (20,0) {$43\bm{\textcolor{supai}{54}\textcolor{supai}{5}\textcolor{supai}{65}}76$};
\node(label2) at (30,0) {$4354\bm{\textcolor{supai}{65}\textcolor{supai}{6}\textcolor{supai}{76}}$};
\node(label2) at (40,0) {$435465\bm{\textcolor{rose}{76}\textcolor{rose}{7}}$};
\end{tikzpicture}
\end{center}
%\vspace{1em}
In type $B$, expressions for braid chains that lack the $s_0$ generator look and function the same as a braid chain in type $A$. The addition of the generator gives us the potential to perform a \alert{long braid} move.

\vspace{1em}
\begin{tikzpicture}[scale=0.7]
Examples of \alert{braid chains} and their corresponding \alert{tracks}.
%
\node (label1) at (0,0) {$\bm{\textcolor{rose}{212}}0102132$};
\node (label2) at (10,0) {$\bm{\textcolor{supai}{121010}}2132$};
\node(label3) at (20,0) {$12\bm{\textcolor{supai}{010121}}32$};
\node(label4) at (30,0) {$12010\bm{\textcolor{supai}{21232}}$};
\node(label5) at (40,0) {$1201021\bm{\textcolor{rose}{323}}$};
\end{tikzpicture}
\end{block}

%---------------------%

\begin{block}{Maximal braid chains}
For a reduced expression $\w$, a consecutive subword is called a \alert{maximal braid chain} if it is a braid chain that is not contained (with respect to position) in any other braid chain in that word.

\vspace{1em}
A factorization into maximal braid chains is called a \alert{braid chain factorization}. It turns out that braid chain factorizations are unique.
\end{block}

%---------------------%

\begin{block}{Theorem}
Suppose $\w$ and $\w'$ are two reduced expressions for $w \in W(A_n)$ or $w \in W(B_n)$ having braid chain factorizations $b_1\mid b_2\mid \cdots\mid b_m$ and $b'_1\mid b'_2\mid\cdots\mid b'_r$, respectively. Then $\w$ and $\w'$ are braid equivalent iff
\vspace{.2em}
\begin{enumerate}
\item $m=r$
\item $b_i$ is braid equivalent to $b'_i$ for $1\leq i\leq m$.
\end{enumerate}
%Moreover, if $\w$ and $\w'$ are braid equivalent, then $\cs(\w)=\cs(\w')$.
\end{block}

%---------------------%

\begin{block}{Theorem}
If $\w$ is a reduced expression for $w \in W(A_n)$ or $w \in W(B_n)$ having braid chain factorization $b_1 \mid b_2 \mid \cdots \mid b_m$ such that each $\ell(b_i)=2k_i-1$ if $b_i$ is a type $A$ braid chain, and $\ell(b_i)=2k_i$ if $b_i$ is a type $B$ braid chain, then
\[
|[\w]_b|=\prod_{i=1}^m k_i.
\]
\end{block}

\end{column}

%%%%%%%%%%%%%%%

\begin{column}{.33\linewidth}

%---------------------%

\begin{block}{Braid graphs of braid classes}
\vspace{.3em}
For a reduced expression $\w$ of $w \in W$ define the \alert{braid graph} $\mathcal{B}([\w]_b)$ for the braid class $[\w]_b$ via:
%\vspace{-.5em}
\begin{itemize}
\item Vertices: reduced expressions in $[\w]_b$
\item Edges: $\w_1$ and $\w_2$ are connected by an edge iff $\w_1$ and $\w_2$ are related via a single braid move
\end{itemize}
A braid graph is simply a \textcolor{limegreen}{green} connected component in the Matsumoto graph.
\end{block}

%---------------------%

\begin{block}{Theorem (Bidari, Ernst, Samz)}

If $\w$ is a reduced expression for $w \in W(A_n)$ or $w \in W(B_n)$ having braid chain factorization $b_1 \mid b_2 \mid \cdots \mid b_m$ such that each $\ell(b_i)=2k_i-1$ or $\ell(b_i)=2k_i$, depending on whether $b_i$ is a type $A$ or type $B$ braid chain, then
\[
\mathcal{B}([\w]_b) =
\begin{tabular}{c}
\includegraphics[scale=2.85]{TheoremGraph.pdf}
%\begin{tikzpicture}[scale=2.5,every circle node/.style={draw, circle, fill,inner sep=1pt}, decoration={brace,amplitude=7}]
%\node [circle] (a) at (2.5,0){};
%\node [circle] (b) at (2.5,-1){};
%\node (dots) at (2.5,-1.4){$\vdots$};
%\node [circle] at (2.5,-2) (c){};
%\node [circle] at (2.5,-3)(d){};
%\path[-] (a) edge[green,ultra thick] (b);
%\path[-] (c) edge[green,ultra thick] (d);
%\draw [decorate] ([xshift=2mm]a.east) --node[right=2mm]{$k_1$} ([xshift=2mm]d.east);
%
%\node at (4,-1.5) {$\Box$};
%
%\node [circle] (e) at (4.5,0){};
%\node [circle] (f) at (4.5,-1){};
%\node (dots) at (4.5,-1.4){$\vdots$};
%\node [circle] at (4.5,-2) (g){};
%\node [circle] at (4.5,-3)(h){};
%\path[-] (e) edge[green,ultra thick] (f);
%\path[-] (g) edge[green,ultra thick] (h);
%\draw [decorate] ([xshift=2mm]e.east) --node[right=2mm]{$k_2$} ([xshift=2mm]h.east);
%
%\node at (6,-1.5) {$\Box$};
%\node at (6.55,-1.5) {$\cdots$};
%\node at (7,-1.5) {$\Box$};
%
%\node [circle] (i) at (7.5,0){};
%\node [circle] (j) at (7.5,-1){};
%\node (dots) at (7.5,-1.4){$\vdots$};
%\node [circle] at (7.5,-2) (k){};
%\node [circle] at (7.5,-3)(l){};
%\path[-] (i) edge[green,ultra thick] (j);
%\path[-] (k) edge[green,ultra thick] (l);
%\draw [decorate] ([xshift=2mm]i.east) --node[right=2mm]{$k_m$} ([xshift=2mm]l.east);
%\end{tikzpicture}
\end{tabular}
\]
\vspace{-.4em}

\end{block}

%---------------------%

\begin{block}{Example}
Consider the reduced expression $\w = 56521201021328$ for $w \in W(B_9)$, which has braid chain factorization:
\[
565\mid 2120102132 \mid 8.
\]
Then $|[\w]_b|=2\cdot 5\cdot 1 =10$ and $\mathcal{B}([\w]_b)$ has 10 vertices.
\begin{figure}
\includegraphics[scale=2.85]{BraidGraphEx1.pdf}
%\begin{tikzpicture}[scale=1.25,every circle node/.style={draw, circle, fill ,inner sep=1pt}]
%\node [circle,] (1) at (0,-3){};
%\node [circle] (2) at (0,-5){};
%\path[-] (1) edge[green,ultra thick]  (2);
%
%\node (x) at (1.5,-4) {$\Box$};
%
%\node [circle] (3) at (3,0){};
%\node [circle] (4) at (3,-2){};
%\node [circle] (5) at (3,-4){};
%\node [circle] (6) at (3,-6){};
%\node [circle] (7) at (3,-8){};
%\path[-] (3) edge[green,ultra thick]  (4);
%\path[-] (4) edge[green,ultra thick]  (5);
%\path[-] (5) edge[green,ultra thick]  (6);
%\path[-] (6) edge[green,ultra thick]  (7);
%
%\node (e) at (4.5,-4) {$=$};
%
%
%\node [circle] (8) at (6,0){};
%\node [circle] (9) at (6,-2){};
%\node [circle] (10) at (6,-4){};
%\node [circle] (11) at (6,-6){};
%\node [circle] (12) at (6,-8){};
%
%\node [circle] (13) at (8,0){};
%\node [circle] (14) at (8,-2){};
%\node [circle] (15) at (8,-4){};
%\node [circle] (16) at (8,-6){};
%\node [circle] (17) at (8,-8){};
%
%%down
%\path[-] (8) edge[green,ultra thick]  (9);
%\path[-] (9) edge[green,ultra thick]  (10);
%\path[-] (10) edge[green,ultra thick]  (11);
%\path[-] (11) edge[green,ultra thick]  (12);
%\path[-] (13) edge[green,ultra thick]  (14);
%\path[-] (14) edge[green,ultra thick]  (15);
%\path[-] (15) edge[green,ultra thick]  (16);
%\path[-] (16) edge[green,ultra thick]  (17);
%
%%across
%\path[-] (8) edge[green,ultra thick]  (13);
%\path[-] (9) edge[green,ultra thick]  (14);
%\path[-] (10) edge[green,ultra thick]  (15);
%\path[-] (11) edge[green,ultra thick]  (16);
%\path[-] (12) edge[green,ultra thick]  (17);
%
%\end{tikzpicture}
\end{figure}

% Additionally, we define $X_{2l-1}$ to be the number of braid chains of length $2l-1$ in our braid chain factorization.
%\vspace{.6em}

\end{block}

%---------------------%

\begin{block}{Example}

%These products of graphs can also result in three dimensional graphs, as is the case when we
Consider the reduced expression $\w = 12143465676$ for $w \in W(A_7)$, which has braid chain factorization:
\[
121 \mid 434 \mid 65676.
\]
Then $|[\w]_b|=2\cdot 2\cdot 3 = 12$ and $\mathcal{B}([\w]_b)$ has 12 vertices.

\begin{figure}
\includegraphics[scale=3.18]{BraidGraphEx2.pdf}
%\begin{tikzpicture}[scale=1.7,every circle node/.style={draw, circle, fill ,inner sep=1pt}]
%\node [circle] (1) at (0,-3){};
%\node [circle] (2) at (0,-5){};
%\path[-] (1) edge[green,ultra thick] (2);
%
%\node (x1) at (1.5,-4) {$\Box$};
%
%\node [circle] (3) at (3,-3){};
%\node [circle] (4) at (3,-5){};
%\path[-] (3) edge[green,ultra thick] (4);
%
%
%\node (x2) at (4.5,-4) {$\Box$};
%
%\node [circle] (5) at (6,-2){};
%\node [circle] (6) at (6,-4){};
%\node [circle] (7) at (6,-6){};
%\path[-] (5) edge[green,ultra thick]  (6);
%\path[-] (6) edge[green,ultra thick]  (7);
%
%\node (e) at (7.5,-4) {$=$};
%
%
%%not dotted
%\node [circle] (8) at (9,-2){};
%\node [circle] (9) at (9,-4){};
%\node [circle] (10) at (9,-6){};
%\path[-] (8) edge[green,ultra thick]  (9);
%\path[-] (9) edge[green,ultra thick]  (10);
%
%\node [circle] (11) at (9.9,-2.5){};
%\node [circle] (12) at (9.9,-4.5){};
%\node [circle] (13) at (9.9,-6.5){};
%\path[-] (11) edge[green,ultra thick] (12);
%\path[-] (12) edge[green,ultra thick]  (13);
%
%\node [circle] (14) at (11,-2){};
%\node [circle] (15) at (11,-4){};
%\node [circle] (16) at (11,-6){};
%\path[-] (14) edge[green,ultra thick] (15);
%\path[-] (15) edge[green,ultra thick]  (16);
%
%\path[-] (8) edge[green,ultra thick]  (11);
%\path[-] (11) edge[green,ultra thick]  (14);
%\path[-] (9) edge[green,ultra thick]  (12);
%\path[-] (12) edge[green,ultra thick]  (15);
%\path[-] (10) edge[green,ultra thick]  (13);
%\path[-] (13) edge[green,ultra thick]  (16);
%
%%back
%\node [circle] (20) at (10.1,-1.5){};
%\node [circle] (21) at (10.1,-3.5){};
%\node [circle] (22) at (10.1,-5.5){};
%%upper square
%\path[-] (8) edge[green,ultra thick]  (20);
%\path[-] (20) edge[green,ultra thick]  (14);
%%dotted lines
%%\path[draw] (20) edge[green,thick] (21);
%%\path[draw] (21) edge[green,thick]  (22);
%
%\path[draw] (20) edge[green,ultra thick] (21);
%\path[draw] (21) edge[green,ultra thick]  (22);
%
%\path[draw] (9) edge[green,ultra thick]  (21);
%\path[draw] (10) edge[green,ultra thick]  (22);
%
%%\path[draw] (20) edge[green,thick]  (14);
%\path[draw] (21) edge[green,ultra thick]  (15);
%\path[draw] (22) edge[green,ultra thick]  (16);
%
%\end{tikzpicture}
\end{figure}
\vspace{-.03em}
\end{block}


%\begin{block}{Conjecture}
%The expanded star reduction graph \manstar$(A_n)$ is ranked.
%\end{block}
%---------------------%
%\begin{block}{Open Questions}
%\begin{enumerate}
%%\item What does the expanded star reduction graph look like for $A_n$ when $n>3$?
%\item How are braid-related heaps related in \manstar$(A_n)$?
%\item Can we classify the star irreducible heaps?
%%\item How many star irreducible heaps are there?
%%\item How large are the commutation classes associated to star irreducible heaps?
%%\item What is the maximum length of a star irreducible heap?
%\item What are the possible degrees of vertices in \manstar$(A_n)$?
%\item The \alert{weight} of a heap is the number of reduced expressions in the corresponding commutation class. What is the structure of the weights? 
%\item Can we classify the maximal heaps in \manstar$(A_n)$?
%\end{enumerate}
%\end{block}
%---------------------%

\end{column}

\end{columns}

\end{frame}

\end{document}